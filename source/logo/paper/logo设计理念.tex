% !TEX program = xelatex
% 智舆 Logo 设计理念文档
\documentclass[12pt,a4paper,fontset=windows]{ctexart}

% 宏包引入
\usepackage{geometry}
\usepackage{graphicx}
\usepackage{xcolor}
\usepackage{tikz}
\usepackage{enumitem}
\usepackage{setspace}
\usepackage{titlesec}
\usepackage{array}
\usepackage{float}

% 页面设置
\geometry{left=2.5cm,right=2.5cm,top=2.5cm,bottom=2.5cm}

% 颜色定义(与Logo配色一致)
\definecolor{neoncyan}{HTML}{06B6D4}
\definecolor{deepblue}{HTML}{0F172A}
\definecolor{neonpurple}{HTML}{A855F7}

% 标题格式
\titleformat{\section}{\heiti\Large\bfseries\color{neoncyan}}{}{0em}{}
\titleformat{\subsection}{\heiti\large\bfseries}{}{0em}{}

% 行间距
\setstretch{1.5}

\begin{document}

% ==================== 标题页 ====================
\begin{titlepage}
    \centering
    \vspace*{2cm}
    
    % Logo图片
    \includegraphics[width=0.8\textwidth]{logo-new.png}
    
    \vspace{1.5cm}
    
    % 标题
    {\heiti\fontsize{28pt}{36pt}\selectfont\color{neoncyan} 智舆}
    
    \vspace{0.3cm}
    
    {\heiti\fontsize{18pt}{24pt}\selectfont AI城市舆情态势监测感知与决策推演系统}
    
    \vspace{1.5cm}
    
    {\heiti\fontsize{22pt}{28pt}\selectfont Logo 设计理念}
    
    \vfill
    
    {\songti\normalsize 2025年}
    
\end{titlepage}

% ==================== 正文 ====================

\section{设计概述}

"智舆"系统的 Logo 设计旨在通过简洁而富有科技感的视觉语言,传达"\textbf{全域感知、智慧决策}"的核心品牌价值。设计融合了\textbf{人工智能 (AI)}、\textbf{城市 (City)} 与\textbf{舆情交流 (Public Opinion)} 三大核心元素,展现了系统作为城市智慧大脑的洞察力与决策力。

\begin{figure}[htbp]
    \centering
    \includegraphics[width=0.4\textwidth]{logo-new.png}
    \caption{智舆系统Logo}
\end{figure}

\section{图形寓意}

\subsection{A. 六边形框架与城市天际线}

Logo 整体呈现\textbf{立体六边形}结构,顶部延伸出\textbf{城市建筑剪影}。

\begin{itemize}[leftmargin=2em]
    \item \textbf{寓意}:六边形象征稳定、高效与互联,是蜂巢结构的抽象,代表系统的模块化与协同能力。城市天际线则直观表达了"AI+城市"的应用场景。
    \item \textbf{关联}:呼应项目中"智慧城市治理"与"3D城市地图"的核心功能。
\end{itemize}

\subsection{B. 负空间对话气泡}

Logo 中心的负空间巧妙地形成了一个\textbf{六边形对话气泡}的轮廓。

\begin{itemize}[leftmargin=2em]
    \item \textbf{寓意}:象征着舆情的核心——"公众的声音"与"社会交流"。同时也隐喻了系统具备的\textbf{自然语言交互}能力,用户可以通过对话与系统进行沟通。
    \item \textbf{关联}:呼应项目中的"语音交互层"(讯飞TTS/ASR)与"舆情分析"本质。
\end{itemize}

\subsection{C. 流动的丝带与数据流}

环绕六边形的\textbf{立体丝带}呈现出流动感,模拟数据在系统中的传输与处理。

\begin{itemize}[leftmargin=2em]
    \item \textbf{寓意}:代表 AI 大模型(讯飞星火)对海量复杂数据的深度分析与逻辑推演,寓意打破信息孤岛,实现数据的智慧互联。
    \item \textbf{关联}:呼应项目中"AI 分析层"与"多Agent协作"的技术架构。
\end{itemize}

\subsection{D. 莫比乌斯环式循环}

丝带的交织形成类似\textbf{莫比乌斯环}的无限循环结构。

\begin{itemize}[leftmargin=2em]
    \item \textbf{寓意}:象征系统对城市舆情态势 24 小时不间断的实时监测与持续迭代优化。
    \item \textbf{关联}:呼应项目中"实时监听"与"走向预测"的技术特性。
\end{itemize}

\section{色彩哲学}

Logo 采用了符合项目"赛博朋克"前端风格的渐变配色方案:

\begin{table}[htbp]
\centering
\renewcommand{\arraystretch}{1.8}
\begin{tabular}{|>{\centering\arraybackslash}p{3cm}|>{\centering\arraybackslash}p{3cm}|p{6cm}|}
\hline
\textbf{色彩} & \textbf{色值} & \textbf{象征意义} \\
\hline
\colorbox{neoncyan}{\textcolor{white}{荧光青}} & \#06B6D4 & 代表敏锐、创新与 AI 的活力,作为主色调传达科技感与未来感 \\
\hline
\colorbox{deepblue}{\textcolor{white}{深邃蓝}} & \#0F172A & 代表理智、大数据与底层算法的严谨,传达系统的可靠性与安全性 \\
\hline
\colorbox{neonpurple}{\textcolor{white}{霓虹紫}} & \#A855F7 & 代表智慧、神秘与 AI 的决策能力,体现完整的数据到决策链路 \\
\hline
\end{tabular}
\end{table}

渐变色从\textbf{青色}过渡到\textbf{蓝紫色},形成强烈的科技感与未来感,与系统前端的赛博朋克 UI 风格保持高度一致。

% \section{字体设计}

% \begin{itemize}[leftmargin=2em]
%     \item \textbf{中文字体}:采用简洁有力的现代黑体变体,笔画硬朗,体现科技产品的专业度与力量感。
%     \item \textbf{英文字体}:使用无衬线字体 (Sans-serif),字间距适中,与图形风格保持一致,展现国际化视野。
% \end{itemize}

\section{品牌愿景}

"智舆" Logo 不仅是一个视觉符号,更是"AI+城市"创新精神的载体。它代表了我们致力于通过前沿人工智能技术,让城市舆情:

\begin{center}
\fcolorbox{neoncyan}{white}{
\parbox{0.8\textwidth}{
\centering
\vspace{0.5em}
{\heiti\Large\color{neoncyan} 看得见 · 读得懂 · 判得准}
\vspace{0.5em}
}
}
\end{center}

辅助政府与企业实现科学决策,守护城市舆论安全。

\end{document}
