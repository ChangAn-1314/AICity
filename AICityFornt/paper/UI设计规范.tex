% !TEX program = xelatex
% 智舆系统 UI 设计规范文档
\documentclass[12pt,a4paper,fontset=windows]{ctexart}

% 宏包引入
\usepackage{geometry}
\usepackage{graphicx}
\usepackage{xcolor}
\usepackage{enumitem}
\usepackage{setspace}
\usepackage{titlesec}
\usepackage{array}
\usepackage{booktabs}
\usepackage{float}
\usepackage{listings}
\usepackage{fancyvrb}

% 页面设置
\geometry{left=2.5cm,right=2.5cm,top=2.5cm,bottom=2.5cm}

% 颜色定义(与系统配色一致)
\definecolor{neoncyan}{HTML}{06B6D4}
\definecolor{deepblue}{HTML}{0F172A}
\definecolor{neonpurple}{HTML}{A855F7}
\definecolor{alertred}{HTML}{EF4444}
\definecolor{successgreen}{HTML}{10B981}
\definecolor{textprimary}{HTML}{F8FAFC}
\definecolor{textsecondary}{HTML}{94A3B8}

% 标题格式
\titleformat{\section}{\heiti\Large\bfseries\color{neoncyan}}{}{0em}{}
\titleformat{\subsection}{\heiti\large\bfseries}{}{0em}{}
\titleformat{\subsubsection}{\heiti\normalsize\bfseries}{}{0em}{}

% 行间距
\setstretch{1.5}

% 代码样式
\lstset{
    basicstyle=\ttfamily\small,
    backgroundcolor=\color{deepblue!10},
    frame=single,
    rulecolor=\color{neoncyan!50},
    breaklines=true
}

\begin{document}

% ==================== 标题页 ====================
\begin{titlepage}
    \centering
    \vspace*{3cm}
    
    % 标题
    {\heiti\fontsize{32pt}{40pt}\selectfont\color{neoncyan} 智舆}
    
    \vspace{0.5cm}
    
    {\heiti\fontsize{16pt}{22pt}\selectfont AI城市舆情态势监测感知与决策推演系统}
    
    \vspace{2cm}
    
    {\heiti\fontsize{24pt}{32pt}\selectfont UI 设计规范}
    
    \vspace{0.5cm}
    
    {\songti\large User Interface Design Specification}
    
    \vfill
    
    {\songti\normalsize 2025年}
    
\end{titlepage}

% ==================== 正文 ====================

\section{设计理念}

\begin{itemize}[leftmargin=2em]
    \item \textbf{核心主题}:Cyberpunk(赛博朋克)+ Glassmorphism(毛玻璃拟态)+ Data Vis(数据可视化)
    \item \textbf{视觉关键词}:沉浸式、半透明、霓虹光效、全息投影、流动感
    \item \textbf{用户体验}:强调信息的层级与直观展示,通过微交互提升操作的反馈感
\end{itemize}

\section{色彩系统}

采用深色模式作为基底,以突显光效和数据图表。

\subsection{背景色}

\begin{table}[H]
\centering
\begin{tabular}{>{\centering\arraybackslash}p{3cm}p{4cm}p{6cm}}
\toprule
\textbf{名称} & \textbf{色值} & \textbf{用途} \\
\midrule
\colorbox{deepblue}{\textcolor{white}{Base}} & \#0F172A (Slate 900) & 深邃蓝黑色,作为主背景 \\
Overlay & rgba(15, 23, 42, 0.6) & 用于遮罩或叠加层 \\
\bottomrule
\end{tabular}
\end{table}

\subsection{主色调与强调色}

\begin{table}[H]
\centering
\begin{tabular}{>{\centering\arraybackslash}p{3cm}p{4cm}p{6cm}}
\toprule
\textbf{名称} & \textbf{色值} & \textbf{用途} \\
\midrule
\colorbox{neoncyan}{\textcolor{white}{Cyber Blue}} & \#06B6D4 (Cyan 500) & 核心数据、选中状态、科技线条 \\
\colorbox{neonpurple}{\textcolor{white}{Neon Purple}} & \#A855F7 (Purple 500) & 辅助图形、渐变光晕 \\
\colorbox{alertred}{\textcolor{white}{Alert Red}} & \#EF4444 (Red 500) & 警报、异常状态 \\
\colorbox{successgreen}{\textcolor{white}{Success Green}} & \#10B981 (Emerald 500) & 健康、通畅状态 \\
\bottomrule
\end{tabular}
\end{table}

\subsection{中性色}

\begin{table}[H]
\centering
\begin{tabular}{>{\centering\arraybackslash}p{3cm}p{4cm}p{6cm}}
\toprule
\textbf{名称} & \textbf{色值} & \textbf{用途} \\
\midrule
Text Primary & \#F8FAFC (Slate 50) & 主要文字 \\
Text Secondary & \#94A3B8 (Slate 400) & 次要标签、说明文字 \\
Glass Border & rgba(255,255,255,0.1) & 卡片描边 \\
\bottomrule
\end{tabular}
\end{table}

\section{布局结构}

采用 \textbf{Bento Grid(便当盒布局)} 或 \textbf{Dashboard(仪表盘布局)} 风格,利用 CSS Grid/Flex 实现全响应式适配。

\subsection{断点策略}

\begin{itemize}[leftmargin=2em]
    \item \textbf{Desktop (lg/xl)}:完整的三栏布局(侧边栏 + 核心地图 + 右侧面板)
    \item \textbf{Mobile (xs/sm)}:单列布局。地图作为背景,功能面板折叠,通过底部抽屉或浮动按钮唤起
\end{itemize}

\subsection{层级结构}

\begin{table}[H]
\centering
\begin{tabular}{>{\centering\arraybackslash}p{3cm}p{4cm}p{6cm}}
\toprule
\textbf{层级} & \textbf{名称} & \textbf{内容} \\
\midrule
Layer 0 & 底层 & 动态背景或3D城市地图模型 \\
Layer 1 & 顶层 & 悬浮的半透明玻璃卡片 \\
Layer 2 & 覆盖层 & 全屏模态(Admin Overlay) \\
\bottomrule
\end{tabular}
\end{table}

\textbf{Layer 1 细分区域}:
\begin{itemize}[leftmargin=2em]
    \item \textbf{顶部 Header}:全局状态栏(时间、天气、系统通知、用户档案)
    \item \textbf{顶侧 Sidebar}:导航菜单(概览、交通、能源、安防、设置)
    \item \textbf{中心 Main View}:核心交互区(3D地图、实时监控画面、主数据大屏)
    \item \textbf{右侧 Widgets}:数据模块(关键指标KPI、实时日志、控制面板)
\end{itemize}

\section{关键组件设计}

\subsection{玻璃面板 (GlassPanel)}

所有面板采用统一的毛玻璃效果,确保背景可见,增强空间感。

\textbf{样式规范}:
\begin{itemize}[leftmargin=2em]
    \item \textbf{背景}:\texttt{bg-slate-900/40}(高模糊度 \texttt{backdrop-blur-xl})
    \item \textbf{边框}:\texttt{border border-white/10}
    \item \textbf{圆角}:\texttt{rounded-2xl} 或 \texttt{rounded-3xl}
    \item \textbf{阴影}:\texttt{shadow-lg shadow-cyan-500/10}(带微弱色光的阴影)
\end{itemize}

\subsection{导航 (Navigation)}

采用顶部 \textbf{DynamicIsland(灵动岛)} 风格导航,整合状态与菜单,节省空间。

\subsection{数据可视化}

\begin{itemize}[leftmargin=2em]
    \item \textbf{仪表盘}:环形进度条,使用青色到紫色的渐变
    \item \textbf{波形图}:模拟声波或心跳,展示实时流量/能耗
    \item \textbf{地图标记}:在3D地图上使用脉冲光点(\texttt{animate-ping})标记事件地点
\end{itemize}

\subsection{字体排版}

\begin{itemize}[leftmargin=2em]
    \item \textbf{字体}:Inter 或 Orbitron(用于标题/数字,增加科技感)
    \item \textbf{数字}:等宽字体(Monospaced),确保数字跳动时布局不抖动
\end{itemize}

\section{交互与动效}

\begin{itemize}[leftmargin=2em]
    \item \textbf{Hover效果}:鼠标悬停卡片时,边框亮度增加,背景轻微提亮,或出现光标跟随效果
    \item \textbf{加载动画}:页面进入时,卡片依次错落上浮(staggered fade-in-up)
    \item \textbf{数据更新}:数字变化时采用滚动计数器动画(Counter up)
    \item \textbf{背景}:缓慢流动的极光或网格线条,营造"城市呼吸"的感觉
\end{itemize}

\section{技术实现}

基于当前采用的技术栈(Vue 3 + Tailwind):

\begin{table}[H]
\centering
\begin{tabular}{p{3cm}p{10cm}}
\toprule
\textbf{模块} & \textbf{技术选型} \\
\midrule
框架 & Vue 3 + Vite(Composition API) \\
布局 & CSS Grid / Flexbox + TailwindCSS \\
样式 & Tailwind CSS(opacity modifiers, backdrop-blur, gradients) \\
UI组件 & Element Plus(企业级组件库) \\
图标 & @element-plus/icons-vue(加发光滤镜) \\
图表 & ECharts(支持3D和词云) \\
3D/地图 & 高德 JS API 2.0 + Three.js(GLCustomLayer) \\
动画 & CSS Transitions + Vue Transition \\
状态管理 & Pinia \\
\bottomrule
\end{tabular}
\end{table}

\end{document}
