% !TEX program = xelatex
% 智舆系统 UX 设计方案文档
\documentclass[12pt,a4paper,fontset=windows]{ctexart}

% 宏包引入
\usepackage{geometry}
\usepackage{graphicx}
\usepackage{xcolor}
\usepackage{enumitem}
\usepackage{setspace}
\usepackage{titlesec}
\usepackage{array}
\usepackage{booktabs}
\usepackage{float}
\usepackage{listings}
\usepackage{fancyvrb}

% 页面设置
\geometry{left=2.5cm,right=2.5cm,top=2.5cm,bottom=2.5cm}

% 颜色定义(与系统配色一致)
\definecolor{neoncyan}{HTML}{06B6D4}
\definecolor{deepblue}{HTML}{0F172A}
\definecolor{neonpurple}{HTML}{A855F7}
\definecolor{alertred}{HTML}{EF4444}
\definecolor{successgreen}{HTML}{10B981}

% 标题格式
\titleformat{\section}{\heiti\Large\bfseries\color{neoncyan}}{}{0em}{}
\titleformat{\subsection}{\heiti\large\bfseries}{}{0em}{}
\titleformat{\subsubsection}{\heiti\normalsize\bfseries}{}{0em}{}

% 行间距
\setstretch{1.5}

% 代码样式
\lstset{
    basicstyle=\ttfamily\small,
    backgroundcolor=\color{deepblue!10},
    frame=single,
    rulecolor=\color{neoncyan!50},
    breaklines=true
}

\begin{document}

% ==================== 标题页 ====================
\begin{titlepage}
    \centering
    \vspace*{3cm}
    
    % 标题
    {\heiti\fontsize{32pt}{40pt}\selectfont\color{neoncyan} 智舆}
    
    \vspace{0.5cm}
    
    {\heiti\fontsize{16pt}{22pt}\selectfont AI城市舆情态势监测感知与决策推演系统}
    
    \vspace{2cm}
    
    {\heiti\fontsize{24pt}{32pt}\selectfont UX 设计方案}
    
    \vspace{0.5cm}
    
    {\songti\large User Experience Design Specification}
    
    \vfill
    
    {\songti\normalsize 2025年}
    
\end{titlepage}

% ==================== 正文 ====================

\section{设计目标}

\begin{itemize}[leftmargin=2em]
    \item \textbf{沉浸式指挥感}:让用户(政府/企业决策者)感觉自己处于一个全知全能的"城市大脑"指挥舱中
    \item \textbf{数据降噪}:在炫酷的视觉下,确保核心舆情数据依然清晰可读,避免视觉干扰
    \item \textbf{闭环决策}:引导用户完成从"发现问题"到"解决问题"的完整闭环
\end{itemize}

\section{用户画像与核心场景}

\subsection{核心用户}

\begin{itemize}[leftmargin=2em]
    \item 政府城市管理者
    \item 企业公关负责人
\end{itemize}

\subsection{核心场景}

\begin{table}[H]
\centering
\begin{tabular}{p{3cm}p{10cm}}
\toprule
\textbf{场景} & \textbf{描述} \\
\midrule
\textbf{平态监视} & 挂在大屏上,实时查看城市舆情热力图,关注异常波动 \\
\textbf{危态处置} & 突发热点事件,需要快速定位地点、还原现场、预测走向,并模拟决策后果 \\
\bottomrule
\end{tabular}
\end{table}

\section{信息架构}

将系统功能映射到布局区域中:

\begin{table}[H]
\centering
\small
\begin{tabular}{p{2.8cm}p{3.5cm}p{6.5cm}}
\toprule
\textbf{功能模块} & \textbf{UI区域} & \textbf{交互形式} \\
\midrule
全局导航/状态 & 顶部灵动岛 & 始终可见,显示系统时间、当前城市、全网情绪指数 \\
\midrule
1. 实时监测 & 左侧Sidebar & 实时滚动的舆情News Ticker,热搜榜单,点击联动地图 \\
\midrule
2. 地图可视化 & 中心Main View & 3D城市模型,热点事件以"光柱"或"脉冲圈"标记 \\
\midrule
3. LLM分析/还原 & 右侧Widget(上) & 关键词云、事件摘要、3D现场还原窗口 \\
\midrule
4. 走向预测 & 右侧Widget(中) & 折线图/波形图,展示"过去-现在-未来"趋势 \\
\midrule
5. 决策模拟 & 右侧Widget(下) & 交互式输入框/滑块,大模型反馈推演结果 \\
\midrule
6. 后台管理 & 全屏覆盖层 & 全屏毛玻璃覆盖层,不跳转页面,保持沉浸感 \\
\bottomrule
\end{tabular}
\end{table}

\section{核心用户体验流程}

\subsection{突发舆情事件处置流程}

\subsubsection{阶段一:感知 (Monitoring)}

\begin{itemize}[leftmargin=2em]
    \item \textbf{视觉}:3D地图上某区域(如信阳市浉河区)出现红色脉冲警报
    \item \textbf{交互}:左侧实时列表弹出"突发"标签的新闻条目
    \item \textbf{操作}:用户点击地图上的红点,或点击列表条目
\end{itemize}

\subsubsection{阶段二:分析 (Analysis \& Reconstruction)}

\begin{itemize}[leftmargin=2em]
    \item \textbf{视觉}:
    \begin{itemize}
        \item 地图视角自动平滑推拉(Camera Zoom)到事发地点
        \item 右侧面板滑出"事件详情卡片"(Glass Card)
        \item 如果可用,在地图上方弹窗展示AI生成的"现场3D还原"全息影像
    \end{itemize}
    \item \textbf{数据}:显示大模型生成的"事件摘要"、"情绪占比"分析
\end{itemize}

\subsubsection{阶段三:预测 (Prediction)}

\begin{itemize}[leftmargin=2em]
    \item \textbf{视觉}:地图上显示动态箭头,预示舆情可能扩散的区域(如周边社区)
    \item \textbf{数据}:趋势图展示未来24小时热度预测曲线
    \item \textbf{AI建议}:界面弹出AI助手建议:"建议立即发布官方通报"
\end{itemize}

\subsubsection{阶段四:模拟与决策 (Simulation \& Decision)}

\begin{itemize}[leftmargin=2em]
    \item \textbf{交互}:用户打开"决策模拟器"(底部或侧边浮层)
    \item \textbf{操作}:用户输入/选择决策方案(例如:选择"发布辟谣公告" vs "冷处理")
    \item \textbf{反馈}:
    \begin{itemize}
        \item 系统即时计算,地图上的红色警报区域根据模拟结果发生变化(变绿或扩散)
        \item 趋势图生成虚线分支,对比不同决策下的未来走向
    \end{itemize}
\end{itemize}

\subsection{系统配置与数据管理 (Admin Flow)}

\begin{itemize}[leftmargin=2em]
    \item \textbf{入口}:用户点击顶部灵动岛的"设置"图标
    \item \textbf{交互}:
    \begin{itemize}
        \item 主界面背景模糊变暗(Backdrop Blur)
        \item "后台管理中心"作为全屏覆盖层(Modal Overlay)平滑淡入
        \item 侧边栏提供Dashboard、热点管理、系统设置等导航
    \end{itemize}
    \item \textbf{退出}:点击关闭按钮或按ESC,覆盖层淡出,无缝回到3D地图指挥界面
\end{itemize}

\section{关键交互细节设计}

\begin{table}[H]
\centering
\begin{tabular}{p{3.5cm}p{9.5cm}}
\toprule
\textbf{交互模式} & \textbf{描述} \\
\midrule
\textbf{聚焦模式} & 点击具体事件时,周围非相关UI自动降低不透明度(Dimming),背景地图模糊化,聚焦当前事件卡片 \\
\midrule
\textbf{数据流光} & 在组件连接处使用流动的光线效果,暗示数据正在实时传输和大模型正在推理 \\
\midrule
\textbf{语音交互} & 考虑到指挥大屏场景,增加语音唤醒AI的提示:"嘿AI,模拟一下方案B的后果" \\
\bottomrule
\end{tabular}
\end{table}

\section{多端适配策略}

\subsection{响应式设计}

采用单一代码库,通过CSS媒体查询适配不同屏幕。

\subsection{Web/PC端(核心)}

\begin{itemize}[leftmargin=2em]
    \item 全功能3D交互
    \item 悬浮面板布局
    \item 支持鼠标精细操作
\end{itemize}

\subsection{移动端 (Mobile)}

\begin{itemize}[leftmargin=2em]
    \item \textbf{主页自适应}:保持与PC端相同的URL(\texttt{/})
    \item \textbf{布局调整}:侧边栏自动收起为汉堡菜单,悬浮面板转为底部抽屉(Bottom Sheet)或全屏卡片
    \item \textbf{交互优化}:针对触摸操作优化点击区域,禁用复杂的3D交互或切换为轻量级地图模式
\end{itemize}

\section{原型结构建议}

建议在\texttt{src}目录下建立以下结构来支撑UX:

\begin{table}[H]
\centering
\small
\begin{tabular}{p{5cm}p{8cm}}
\toprule
\textbf{目录/文件} & \textbf{功能} \\
\midrule
\texttt{components/layout/} & 布局组件 \\
\quad AppShell.vue & 整体布局容器(包含背景、Header) \\
\quad DynamicIsland.vue & 灵动岛导航 \\
\midrule
\texttt{components/features/Map/} & 地图功能 \\
\quad CityMap3D-AMap.vue & 核心地图组件(高德3D) \\
\midrule
\texttt{components/features/Monitor/} & 监测功能 \\
\quad NewsTicker.vue & 左侧实时舆情流 \\
\quad HotspotDetail.vue & 舆情详情面板 \\
\quad FilterPanel.vue & 筛选面板 \\
\midrule
\texttt{components/features/Analysis/} & 分析功能 \\
\quad InsightCard.vue & 右侧AI分析卡片 \\
\quad KeywordCloud.vue & 关键词云 \\
\quad TrendChart.vue & 趋势图表 \\
\midrule
\texttt{components/features/Simulation/} & 模拟功能 \\
\quad DecisionPanel.vue & 决策模拟控制台 \\
\quad SimulationResult.vue & 模拟结果展示 \\
\midrule
\texttt{components/features/Voice/} & 语音功能 \\
\quad VoiceButton.vue & 语音交互按钮 \\
\midrule
\texttt{components/ui/} & 基础UI组件 \\
\quad GlassPanel.vue & 基础玻璃容器 \\
\quad NeonButton.vue & 交互按钮 \\
\midrule
\texttt{stores/} & 状态管理 \\
\quad sentiment.js & 舆情状态管理 \\
\quad map.js & 地图状态管理 \\
\quad voice.js & 语音状态管理 \\
\midrule
\texttt{api/} & API接口 \\
\quad sentiment.js & 舆情API \\
\quad decision.js & 决策API \\
\midrule
\texttt{services/} & 服务层 \\
\quad websocket.js & WebSocket服务 \\
\bottomrule
\end{tabular}
\end{table}

\section{设计总结}

本UX方案将抽象的"分析、预测、模拟"转化为具体可视化的操作步骤,核心理念如下:

\begin{center}
\fcolorbox{neoncyan}{white}{
\parbox{0.85\textwidth}{
\centering
\vspace{0.5em}
{\heiti\Large\color{neoncyan} 感知 $\rightarrow$ 分析 $\rightarrow$ 预测 $\rightarrow$ 决策}

\vspace{0.3em}
{\songti 让城市管理者在沉浸式指挥舱中完成舆情闭环处置}
\vspace{0.5em}
}
}
\end{center}

\end{document}
